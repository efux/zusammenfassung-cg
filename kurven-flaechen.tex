\chapter{Kurven und Flächen}

\section{Kurven in der Ebene}
Gar nicht so schwierig! Wir haben irgendein Intervall, also eine Zahlenreihe von a nach b. Mathematisch ausgedrückt wäre das dann
\begin{displaymath}
[a,b]\rightarrow\mathbb{R}
\end{displaymath}
All diese Zahlen in diesem Intervall setzen wir einfach in eine Funktion hinein und voila - eine Kurve. Jetzt gibt es eine implizite Art, eine Kurve darzustellen und eine explizite Art.
\subsection{Implizite Darstellung}
Am Beispiel eines Kreises wäre die implizite Darstellung:
\begin{displaymath}
x^2 + y^2 - r^2 = 0
\end{displaymath}
Das heisst in einer impliziten Darstellung können wir z.B. nicht einfach die Zahlen von a nach b einsetzen, sondern die Gleichung ist einfach für alle zutreffenden Punkte = 0. Wenn wir Glück haben, ist die implizite Darstellung umwandelbar in eine explizite.
\subsection{Explizite Darstellung}
Eigentlich einfach eine Funktion.
Für den oberen Halbkreis:
\begin{displaymath}
y = \sqrt{r^2-x^2}
\end{displaymath}
und den unteren Halbkreis:
\begin{displaymath}
y = -\sqrt{r^2-x^2}
\end{displaymath}
\subsection{Parameter Darstellung}
\(t\) bedeutet hier die \textit{Zeit} und geht von a nach b. Beispiel mit dem Kreis, wo die \textit{Zeit} von \(0\) bis \(2\pi\) geht:
\begin{displaymath}
X(t) = 
\begin{pmatrix}
x_1(t) \\
x_2(t)
\end{pmatrix} = 
\begin{pmatrix}
r*cos(t)\\
r*sin(t)
\end{pmatrix}
\end{displaymath}
Die Parameterdarstellung ist nicht unbedingt eindeutig, also auch z.B. 
\begin{displaymath}
X(t) = 
\begin{pmatrix}
r*cos(2t)\\
r*sin(2t)
\end{pmatrix}
\end{displaymath}
beschreibt einen Kreis, wobei hier die \textit{Zeit} von 0 bis \(\pi\) geht - also er wird einfach 'schneller' durchlaufen.